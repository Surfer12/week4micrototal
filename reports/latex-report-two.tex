\documentclass[12pt]{article}
\usepackage[utf8]{inputenc}
\usepackage[T1]{fontenc}
\usepackage{amsmath, amssymb, amsfonts}
\usepackage{graphicx}
\usepackage{listings}
\usepackage{xcolor}
\usepackage{hyperref}
\usepackage{booktabs}
\usepackage{geometry}
\geometry{margin=1in}

\title{Lab Two: Building with Gates (Carry-In = 0)}
\author{Due Date: Monday, 2/3 \\ Week 3 (23, 25) – CIS 240 Micro Arch \& Prog (31563)}
\date{}

\lstset{
  backgroundcolor=\color{white},
  basicstyle=\footnotesize\ttfamily,
  breaklines=true,
  captionpos=b,
  commentstyle=\color{green},
  keywordstyle=\color{blue},
  stringstyle=\color{black},
  numbers=none,
  frame=single,
}

\begin{document}

\maketitle

\section{Introduction}
\subsection{Objective}
\textit{"The objective of this lab is to design, simulate, and test a single-bit adder block with Carry-In = 0 using NAND gates and other basic logic in LTSpice. Then, we optimize the Least Significant Bit (LSB) block by removing the carry-in input entirely, creating a simplified half-adder."}

\subsection{Overview}
In this lab report, we undertake the following:
\begin{itemize}
    \item \textbf{Part 1:} Implement a single-bit adder assuming Carry-In = 0.
    \item \textbf{Part 2:} Optimize the design by removing the Carry-In input from the LSB block.
\end{itemize}
Verification is achieved through truth tables, SOP derivations, LTSpice simulations, and integration testing.

\section{Part 1: Main Block Design (Carry-In = 0)}
\subsection{Understanding the Problem}
Design a half-adder that adds two input bits, A[0] and B[0], with a fixed carry-in of 0.

\subsection{Truth Tables}
\subsubsection*{SUM Truth Table}
\begin{center}
\begin{tabular}{|c|c|c|l|}
\hline
\textbf{A[0]} & \textbf{B[0]} & \textbf{Sum[0]} & \textbf{Explanation} \\ \hline
0 & 0 & 0 & 0 + 0 = 0 (no carry) \\ \hline
0 & 1 & 1 & 0 + 1 = 1 (no carry) \\ \hline
1 & 0 & 1 & 1 + 0 = 1 (no carry) \\ \hline
1 & 1 & 0 & 1 + 1 = 2, thus 0 with a carry of 1 \\ \hline
\end{tabular}
\end{center}

\subsubsection*{CARRY-OUT Truth Table}
\begin{center}
\begin{tabular}{|c|c|c|l|}
\hline
\textbf{A[0]} & \textbf{B[0]} & \textbf{Carry\_Out} & \textbf{Explanation} \\ \hline
0 & 0 & 0 & No carry generated \\ \hline
0 & 1 & 0 & 0 + 1 = 1 (no carry) \\ \hline
1 & 0 & 0 & 1 + 0 = 1 (no carry) \\ \hline
1 & 1 & 1 & 1 + 1 = 2, carry generated \\ \hline
\end{tabular}
\end{center}

\noindent \textbf{Block Function:}  
\[
\text{Sum}[0] = A[0] \oplus B[0], \quad \text{Carry\_Out} = A[0] \cdot B[0]
\]
The design behaves as a half-adder since Carry-In is fixed at 0.

\subsection{Derivation of Logic (SOP Method)}
\noindent \textbf{Sum[0]:}
\[
\text{Sum}[0] = \overline{A[0]}\,B[0] + A[0]\,\overline{B[0]} \quad \text{(i.e., } A[0] \oplus B[0] \text{)}
\]
\noindent \textbf{Carry\_Out:}
\[
\text{Carry\_Out} = A[0] \cdot B[0]
\]

\subsection{Circuit Implementation Steps}
\begin{enumerate}
    \item Enter the circuit in LTSpice.
    \item Create a block symbol for the half-adder.
    \item Develop a test schematic by applying pulse/voltage sources.
    \item Perform transient simulation to verify the truth tables.
    \item Backup all design files (e.g., \texttt{.asc}, \texttt{.asy}) and simulation screenshots.
\end{enumerate}

\section{Part 2: LSB Block Optimization (No Carry-In Input)}
\subsection{Schematic Layout}
\begin{itemize}
    \item \textbf{Inputs:} Label as \texttt{A[0]} and \texttt{B[0]}.
    \item \textbf{XOR Implementation:} Use two NAND gates with corresponding inverters to generate the XOR function:
    \begin{itemize}
        \item One NAND gate for \((A, \overline{B})\).
        \item One NAND gate for \((\overline{A}, B)\).
        \item Combine these outputs with an OR or NAND-inverter configuration to yield \texttt{Sum[0]}.
    \end{itemize}
    \item \textbf{Carry Logic:} Implement using an AND gate:
    \[
    \text{Carry\_Out} = A[0] \cdot B[0]
    \]
\end{itemize}

\subsection{Rationale for Using NAND + Inverters}
NAND gates are universal. Inverters allow for the creation of XOR functionality with fewer parts and simplified wiring, especially when the Carry-In is omitted.

\subsection{Symbol Creation in LTSpice}
\begin{enumerate}
    \item Open the symbol editor (\texttt{Hierarchy $\rightarrow$ Open Symbol} or \texttt{Edit $\rightarrow$ Create Symbol}).
    \item Draw the symbol and add the pins: \texttt{A[0]}, \texttt{B[0]}, \texttt{Sum[0]}, and \texttt{Carry\_Out}.
    \item Save the symbol as a \texttt{.asy} file.
\end{enumerate}

\subsection{Top-Level Test Schematic \& Simulation}
\begin{enumerate}
    \item Place the new symbol as a subcircuit.
    \item Drive inputs with test patterns (00, 01, 10, 11) using pulse/voltage sources.
    \item Connect waveform probes to monitor \texttt{Sum[0]} and \texttt{Carry\_Out}.
    \item Ensure that simulation outputs match the predetermined truth tables.
\end{enumerate}

\subsection{Integration}
\begin{enumerate}
    \item Remove the old LSB symbol.
    \item Insert the optimized LSB block.
    \item Re-run simulations to confirm valid behavior across all test conditions.
\end{enumerate}

\section{Discussion}
Discuss the following:
\begin{itemize}
    \item Comparison between the original design and the optimized LSB block.
    \item Challenges encountered while converting NAND gates to perform XOR functionality.
    \item How the half-adder functionality can be extended into multi-bit adder circuits.
\end{itemize}

\section{Conclusion}
\textbf{Summary:}  
"Through SOP derivations and LTSpice simulations, we confirmed that both the main block (with Carry-In = 0) and the optimized LSB design (with no Carry-In) correctly implement the half-adder behavior."

\noindent \textbf{Key Difference:}  
The optimized design omits the Carry-In, simplifying the circuitry while maintaining proper function.

\section{Next Steps}
\begin{itemize}
    \item Extend the design into a complete 4-bit or n-bit ripple-carry adder.
    \item Further optimize timing, power, and physical layout.
\end{itemize}

\section{Appendix and References}
\begin{itemize}
    \item \textbf{Truth Tables and SOP Derivations:} See Sections 2.2 and 2.3.
    \item \textbf{LTSpice Files:} Include \texttt{.asc} schematics, \texttt{.asy} symbols, and simulation screenshots.
    \item \textbf{References:} List textbooks, lecture materials, and online resources consulted.
\end{itemize}

\end{document} 